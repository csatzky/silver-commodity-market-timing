% Options for packages loaded elsewhere
\PassOptionsToPackage{unicode}{hyperref}
\PassOptionsToPackage{hyphens}{url}
\PassOptionsToPackage{dvipsnames,svgnames*,x11names*}{xcolor}
%
\documentclass[
  12pt,
]{article}
\usepackage[sc, osf]{mathpazo}
\usepackage{amssymb,amsmath}
\usepackage{ifxetex,ifluatex}
\ifnum 0\ifxetex 1\fi\ifluatex 1\fi=0 % if pdftex
  \usepackage[T1]{fontenc}
  \usepackage[utf8]{inputenc}
  \usepackage{textcomp} % provide euro and other symbols
\else % if luatex or xetex
  \usepackage{unicode-math}
  \defaultfontfeatures{Scale=MatchLowercase}
  \defaultfontfeatures[\rmfamily]{Ligatures=TeX,Scale=1}
\fi
% Use upquote if available, for straight quotes in verbatim environments
\IfFileExists{upquote.sty}{\usepackage{upquote}}{}
\IfFileExists{microtype.sty}{% use microtype if available
  \usepackage[]{microtype}
  \UseMicrotypeSet[protrusion]{basicmath} % disable protrusion for tt fonts
}{}
\makeatletter
\@ifundefined{KOMAClassName}{% if non-KOMA class
  \IfFileExists{parskip.sty}{%
    \usepackage{parskip}
  }{% else
    \setlength{\parindent}{0pt}
    \setlength{\parskip}{6pt plus 2pt minus 1pt}}
}{% if KOMA class
  \KOMAoptions{parskip=half}}
\makeatother
\usepackage{xcolor}
\IfFileExists{xurl.sty}{\usepackage{xurl}}{} % add URL line breaks if available
\IfFileExists{bookmark.sty}{\usepackage{bookmark}}{\usepackage{hyperref}}
\hypersetup{
  colorlinks=true,
  linkcolor=blue,
  filecolor=Maroon,
  citecolor=Blue,
  urlcolor=blue,
  pdfcreator={LaTeX via pandoc}}
\urlstyle{same} % disable monospaced font for URLs
\usepackage[margin=1in]{geometry}
\usepackage{graphicx,grffile}
\makeatletter
\def\maxwidth{\ifdim\Gin@nat@width>\linewidth\linewidth\else\Gin@nat@width\fi}
\def\maxheight{\ifdim\Gin@nat@height>\textheight\textheight\else\Gin@nat@height\fi}
\makeatother
% Scale images if necessary, so that they will not overflow the page
% margins by default, and it is still possible to overwrite the defaults
% using explicit options in \includegraphics[width, height, ...]{}
\setkeys{Gin}{width=\maxwidth,height=\maxheight,keepaspectratio}
% Set default figure placement to htbp
\makeatletter
\def\fps@figure{htbp}
\makeatother
\setlength{\emergencystretch}{3em} % prevent overfull lines
\providecommand{\tightlist}{%
  \setlength{\itemsep}{0pt}\setlength{\parskip}{0pt}}
\setcounter{secnumdepth}{5}
\newcommand{\bquote}{\begin{quote}}
\newcommand{\equte}{\end{quote}}
\newcommand{\cc}{\centering}
\newcommand{\bdquote}{\begin{displayquote}}
\newcommand{\edquote}{\end{displayquote}}
\newcommand{\txtq}{\end{textquote}}
\usepackage{csquotes}
\usepackage{textcomp}
\usepackage{fontawesome5}
\newcommand{\inlinecode}{\texttt}
\newcommand{\typelatex}{\LaTeX}
\newcommand{\vspaceone}{\vspace{1 mm}}
\usepackage{booktabs}
\usepackage{longtable}
\usepackage{array}
\usepackage{multirow}
\usepackage{wrapfig}
\usepackage{float}
\usepackage{colortbl}
\usepackage{pdflscape}
\usepackage{tabu}
\usepackage{threeparttable}
\usepackage{threeparttablex}
\usepackage[normalem]{ulem}
\usepackage{makecell}
\usepackage{xcolor}

\title{Silver Commodity Market Timing--\\
A Simple Macroeconomic Model}
\author{By \href{https://www.linkedin.com/in/christian-satzky/}{Christian
Satzky, FRM}\\
Email: \href{mailto:c.satzky@gmail.com}{\nolinkurl{c.satzky@gmail.com}}}
\date{Last update: 2 May 2021}

\begin{document}
\maketitle

\newpage
\hypersetup{linkcolor=black}
\tableofcontents
\newpage
\hypersetup{linkcolor=blue}

\hypertarget{introduction}{%
\section{Introduction}\label{introduction}}

The silver commodity is a long-established constituent of the precious
metals market. It typically provides investors with greater volatility
than what is commonly observed in the gold market. In this article, I
investigate the determinants of the spot silver price and provide a
valuable tool for industry practitioners.

In the following, I first present a simple, theoretical pricing model.
Secondly, I am empirically investigating key properties of this model by
using recent market data. From these findings, I finally construct a
statistical fair-value indicator for the silver spot price. This
indicator conveniently absorbs statistically significant macroeconomic
information that is relevant to silver spot pricing.

All results are fully reproducible. All data and code are available on
my
\href{https://github.com/csatzky/silver-commodity-market-timing}{GitHub
repo}.\footnote{URL:
  \url{https://github.com/csatzky/silver-commodity-market-timing}}

\hypertarget{silver-spot-pricing-hypothesis}{%
\section{Silver Spot Pricing
Hypothesis}\label{silver-spot-pricing-hypothesis}}

The silver commodity (spot symbol: \texttt{\^{}XAGUSD}) can be thought
of in terms of a zero-coupon paying bond with an infinite time to
maturity. Assuming cashing-out within finite time, I am interested in
the asset's spot price at the end of the arbitrary holding period,
\(T\).

In most simplistic terms, any physical asset appreciates/depreciates in
nominal value with inflation/deflation. It follows that the future
silver spot price, \(P_T\), is a function of the spot price at the
beginning of the investment period, \(P_0\), and USD inflation, \(i\).
In addition, there might be some \(n\) other dependencies, which might
be observable or not, and where it is unknown whether these have a
positive or negative impact on \(P_T\). These are included in the vector
\(X_n\), \(X_n = [x_1, x_2,..., x_n]\).

\[P_{T} = f(P_{0}, i, X_n)\]

Assuming continuous-time, we can discount the future spot price,
\(P_T\), to the current spot price, \(P_t\), using the risk-free USD
interest rate \(r\).\footnote{This implicitly assumes risk-neutrality,
  which further assumes an arbitrage-free market.}

\begin{align}
P_t = \frac{f(P_{0}, i, X_n)}{e^{r(T-t)}}
\end{align}

For illustration purposes, let's assume that
\(f(P_0, i, X_n) = P_0 e^{i(T-t)}\), thereby setting the parameters for
the vector of unknown variables \(X_n\) to zero.

\[P_t = \frac{P_0 e^{i(T-t)}}{e^{r(T-t)}}\]

The partial derivatives with respect to inflation, \(i\), and the
risk-free USD interest rate, \(r\), yield:

\[\frac{\partial P_{t}}{\partial i} = \frac{(T-t)P_0 e^{i(T-t)}}{e^{r(T-t)}}\]

\[\frac{\partial P_{t}}{\partial r} = \frac{(t-T)P_0 e^{i(T-t)}}{e^{r(T-t)}}\]

Note that \(P_0, t, T > 0\) and \(T>t\). Thus,

\begin{align}
\textbf{1.} \hspace{5mm} \frac{\partial P_{t}}{\partial i} &> 0 \notag\\
\textbf{2.} \hspace{5mm} \frac{\partial P_{t}}{\partial r} &< 0 \notag
\end{align}

Hence, I have established that,

\begin{enumerate}
\def\labelenumi{\arabic{enumi}.}
\tightlist
\item
  The silver spot price, \(P_t\), \emph{increases} with rising USD
  inflation, \(i\)
\item
  The silver spot price, \(P_t\), \emph{decreases} with rising risk-free
  USD rates, \(r\).
\end{enumerate}

\vspace{5 mm}

\textbf{\emph{A Word of Caution}}

Let's denote \(s\) the \emph{spread} between USD inflation and the USD
risk-free rate, \(s = i - r\). It is \emph{tempting} to say, whenever
the \emph{change} in the inflation-rate spread, \(s\), is
\emph{positive}, \(P_t\) is increasing. In other words, whenever the
change in inflation is greater than the change in the interest rate, the
silver price, \(P_t\) is rising:

\[\Delta s >0 \implies \Delta P_t > 0\]

However, it is important to understand that we cannot make such claim.
Remember that the exact form of \(f(P_0, i, X_n)\) in equation (1) is
\emph{unknown}. Hence, we cannot infer that \(\Delta i\) and
\(\Delta r\) have an \emph{equal} impact on \(P_t\). However, the exact
impact can be estimated using data, which is done in the next section.

\newpage

\hypertarget{empirical-validation}{%
\section{Empirical Validation}\label{empirical-validation}}

In the following, I am empirically investigating the
\protect\hyperlink{silver-spot-pricing-hypothesis}{silver spot pricing
hypthesis} from the previous section. Hence, it is necessary to find
real-world proxies for USD inflation and USD risk-free rate.

When it comes to inflation, I am using the 10-year Breakeven Inflation
Rate by the Federal Reserve Bank of St.~Louis.\footnote{Federal Reserve
  Bank of St.~Louis, 10-Year Breakeven Inflation Rate {[}T5YIE{]},
  retrieved from FRED, Federal Reserve Bank of St.~Louis;
  \url{https://fred.stlouisfed.org/series/T5YIE}, April 9, 2021.} At
each point in time, this rate represents the \emph{expected inflation}
over the next 10 years. Hence, a \emph{change} in the Breakeven
Inflation Rate can be interpreted as a change in expected inflation,
which is a stationary, \(I(0)\) time-series variable. This rate is
derived from 10-year US Treasury bonds with and without inflation
protection.

When it comes to the risk-free rate, I am using 10-year US treasury bond
rates, which are not indexed for inflation.\footnote{Board of Governors
  of the Federal Reserve System (US), 10-Year Treasury Constant Maturity
  Rate {[}DGS10{]}, retrieved from FRED, Federal Reserve Bank of
  St.~Louis; \url{https://fred.stlouisfed.org/series/DGS10}, April 9,
  2021.}

This selection implies a finite-time holding horizon of 10-years for
this commodity.

For daily data of the last 3 years (from 1 April 2018 to 31 March 2021),
I am fitting the following linear regression model:

\begin{align}
\Delta P_t = \hat\beta^{_i} \Delta i_t + \hat\beta^{_r} \Delta r_t + \epsilon_t
\end{align}

\begin{footnotesize}
Where:

\begin{tabular}{ll}

$P_t$ & Daily XAGUSD exchange rate\\
$\Delta P_t$ & Continuously compounded return of $\scriptstyle P_t$, $\scriptstyle ln \left( \frac{P_t}{P_{t-1}} \right)$\\
$\Delta i_t$ & Change in the 10-year Breakeven Inflation rate, $\scriptstyle i_t - i_{t-1}$\\
$\Delta r_t$ & Change in the 10-year Treasury rate, $\scriptstyle r_t - r_{t-1}$\\
$\epsilon_t$ & Error term, $E[\epsilon_t] = 0$\\

\end{tabular}
\end{footnotesize}

Note that all variables \(\Delta P_t\), \(\Delta i_t\), and
\(\Delta r_t\) are \emph{stationary}, \(I(0)\) time-series. Hence, the
linear regression results below do not suffer from spurious regression,
and t-values, \(R^2\), and \(\beta\) estimates are reliable and
distributed as expected.

\begin{table}[!h]

\caption{\label{tab:lm-fit}Linear Regression Results}
\centering
\begin{tabular}[t]{lrrrr}
\toprule
  & $\hat{\beta}$ & $\textrm{SE}(\hat{\beta})$ & $\textrm{t-value}$ & $\textrm{p-value}$\\
\midrule
$\Delta i_t$ & 0.1999 & 0.0235 & 8.5240 & <0.001\\
$\Delta r_t$ & -0.1017 & 0.0165 & -6.1752 & <0.001\\
\bottomrule
\multicolumn{5}{l}{\rule{0pt}{1em}$\textit{Adj Rsq: 0.0961}$}\\
\end{tabular}
\end{table}

The linear regression model in (2) explains roughly 10\% of the total
variation in the Silver spot price return. In terms of regression in
differences, this is a remarkably convincing result. Using the linear
regression output, we can use statistical theory to (in-)validate the
\protect\hyperlink{silver-spot-pricing-hypothesis}{hypotheses} made in
the previous section. For the \emph{true} value of \(\beta\), we are
interested whether \(\beta^{_i} > 0\) and \(\beta^{_r} < 0\). These
one-sided Frequentist hypothesis tests are carried out below.

\textbf{\emph{Hypothesis Test for Inflation-Beta}}

\begin{align}
\mathbf{H_0}: \hspace{5mm} \beta^{_i} &\leq 0 \notag\\
\mathbf{H_1}: \hspace{5mm} \beta^{_i} &> 0 \notag
\end{align}

\[P[ \textrm{data} | \beta^{_i} \leq 0] = P[T_{\textrm{df=}752} > 8.52] \mathbf{< 0.001}\]

Given \(H_0\) is true, the probability of observing this data (or more
extreme) is \(< 0.001\). Hence, I reject \(H_0\) for \emph{all common
levels of statistical significance} and accept that \(\beta^{_i} > 0\).

\vspace{5 mm}

\textbf{\emph{Hypothesis Test for Risk-Free Rate-Beta}}

\begin{align}
\mathbf{H_0}: \hspace{5mm} \beta^{_r} &\geq 0 \notag\\
\mathbf{H_1}: \hspace{5mm} \beta^{_r} &< 0 \notag
\end{align}

\[P[ \textrm{data} | \beta^{_r} \geq 0] = P[T_{\textrm{df=}752} < -6.18] \mathbf{< 0.001}\]

Given \(H_0\) is true, the probability of observing this data (or more
extreme) is \(< 0.001\). Hence, I reject \(H_0\) for \emph{all common
levels of statistical significance} and accept that \(\beta^{_r} < 0\).

The results of the hypothesis tests provide strong statistical evidence
that the two statements derived from the partial derivatives
\protect\hyperlink{silver-spot-pricing-hypothesis}{in the previous
section} are true.

In summary, I found that;

\begin{enumerate}
\def\labelenumi{\arabic{enumi}.}
\tightlist
\item
  The silver price \emph{increases} with rising expected inflation. This
  is backed by strong statistical evidence.
\item
  The silver price \emph{decreases} with rising 10Y US treasury rates.
  This is backed by strong statistical evidence.
\item
  The silver price is estimated to be \emph{twice as sensitive} to
  changes in inflation than it is to changes in US treasury rates.
\end{enumerate}

\newpage

\hypertarget{construction-of-a-fair-value-silver-indicator}{%
\section{Construction of a Fair-Value Silver
Indicator}\label{construction-of-a-fair-value-silver-indicator}}

According to the linear regression model of the previous section;

\[\widehat{\Delta P_t} = 0.2 \Delta i_t -0.1 \Delta r_t\].

Hence,

\[\mathbb{E}[0.2 \Delta i - 0.1 \Delta r - \Delta P_t] = 0\]

The equation within the expectation can be used to construct a
fundamental Silver spot price indicator. The rolling-time period,
\(\Delta\), can be arbitrarily chosen, e.g.~15 trading days (resembling
3 weeks). The daily indicator value is then computed as
follows:\footnote{Note that continuously compounded returns are additive}

\[0.2 [i_t-i_{t-15}] - 0.1[r_t - r_{t-15}] - ln \left( \frac{P_t}{P_{t-15}} \right)\]

It might be convenient to think of the indicator's values in terms of
simple compounding return. Hence the indicator can be transformed to
reflect that:

\[\text{I}^S_t=100\left( \exp \left[ 0.2 (i_t-i_{t-15}) - 0.1(r_t - r_{t-15}) - ln \left( \frac{P_t}{P_{t-15}} \right) \right] -1 \right)\]

The Silver Indicator, \(I^S_t\), can be interpreted as the
\emph{fundamentally} expected percentage return over the next 3 weeks:

\begin{itemize}
\tightlist
\item
  \(I^S_t > 0\): The current silver spot price is \emph{fundamentally
  undervalued}
\item
  \(I^S_t < 0\): The current silver spot price is \emph{fundamentally
  overvalued}
\item
  \(I^S_t = 0\): The current silver spot price trades at fair value
\end{itemize}

\vspace{5 mm}

\textbf{\emph{Limitations}}

Acknowledge that the \(I^S_t\) is \emph{only} based on developments of
the USD risk-free rate and the expected USD inflation. There is
\protect\hyperlink{empirical-validation}{reasonable statistical
evidence} to use these two macroeconomic variables. Any other potential
factors (e.g.~sentiment, other macroeconomic variables, storage cost,
etc) are neglected. This is due to a lack of data availability and/or a
lack of sensible theoretical considerations to include such additional
variables.

Note that whenever \(I^S_t \neq 0\), \(I^S_t\) \emph{must} eventually
revert to \(0\). This happens because of any of the following reasons:

\begin{enumerate}
\def\labelenumi{\alph{enumi}.}
\tightlist
\item
  The silver spot price will move such that \(I^S_t \to 0\), yielding a
  statistical arbitrage opportunity in the silver market
\item
  The \emph{future} expected inflation will move such that
  \(I^S_t \to 0\)
\item
  The \emph{future} risk-free USD rate will move such that
  \(I^S_t \to 0\)
\item
  The market fails to properly reflect macroeconomic changes within the
  3-week rolling indicator window
\item
  A combination of any of the above
\end{enumerate}

Regardless of these limitations, the indicator performs well in
explaining the silver spot price movements from 2019--2021. Below is a
plot of the silver spot price (\texttt{\^{}XAGUSD}) against \(I^S_t\).

\includegraphics{silver-market-timing_files/figure-latex/silver-indicator-1.pdf}

Observe that,

\begin{enumerate}
\def\labelenumi{\arabic{enumi}.}
\tightlist
\item
  \(I^S_t\) is mean-reverting around zero
\item
  Whenever \(I^S_t\) diverges significantly away from zero, this
  indicates strong buy/sell signals, which in part materialize in
  subsequent \texttt{\^{}XAG/USD} movements
\end{enumerate}

Hence, the \(I^S_t\) indicator can be used to absorb and compare the
information given by the most crucial macroeconomic variables versus the
actual price changes.

\newpage

\hypertarget{silver-market-timing-using-is_t}{%
\subsection{\texorpdfstring{Silver Market Timing Using
\(I^S_t\)}{Silver Market Timing Using I\^{}S\_t}}\label{silver-market-timing-using-is_t}}

To avoid signals in the \(I^S_t\) that are due to noise, let's
arbitrarily define a fundamental buy/sell signal whenever
\(|I^S_t| \ge 10\).

In the plot below, all values \(I^S_t \leq -10\) are highlighted in red
(i.e.~\emph{sell} signal), and values \(I^S_t \geq 10\) are highlighted
in green (i.e.~\emph{buy} signal).

\includegraphics{silver-market-timing_files/figure-latex/highlighted-plot-1.pdf}

Remember that this indicator signals diversions from the estimated
\emph{macroeconomically justifiable} fair-value price. For greater
detail, in the \protect\hyperlink{appendix-zoomed-plots}{appendix} I am
providing zoomed-in plots over specific time periods.

\hypertarget{a-note-on-crypto-currency}{%
\section{A Note on Crypto-Currency}\label{a-note-on-crypto-currency}}

Below is a plot depicting the 10-year US expected inflation\footnote{Federal
  Reserve Bank of St.~Louis, 10-Year Breakeven Inflation Rate
  {[}T5YIE{]}, retrieved from FRED, Federal Reserve Bank of St.~Louis;
  \url{https://fred.stlouisfed.org/series/T5YIE}, April 9, 2021.} (top),
the Bitcoin spot price, \texttt{BTC/USD} (middle), and the silver spot
price, \texttt{XAG/USD} (bottom) from January 2020 to December 2020.

\includegraphics{silver-market-timing_files/figure-latex/bitcoin-1.pdf}
From January 2020 to November 2020, \texttt{BTC/USD} and
\texttt{XAG/USD} shared similar price developments. Both assets seemed
to be driven by a rise in expected inflation. It can be inferred that,
similar to silver, \texttt{BTC/USD} provides inflation protection. On
these terms, \texttt{BTC/USD} can be considered a \emph{low transaction
cost} and \emph{zero storage cost} competitor to the silver commodity.

However, after October 2020, \texttt{BTC/USD} appreciated greatly
without macroeconomic justification. More specifically, \texttt{BTC/USD}
rose from USD 11,736 on 19 October 2020 to roughly USD 30,000 by the end
of the year. On 12 April 2021, it reached more than USD 60,000.
Simultaneously, the overall crypto currency market had an increase in
market capitalization proportionally to the \texttt{BTC/USD}
appreciation.

\textbf{Opinion:} The risk for the silver commodity is that some of it's
potential market capitalization for inflation-protection might be
substituted for investment in cryptocurrency. Hence, further gain in
acceptance and trust in cryptocurrency for the purpose of preserving
value might, in part, drive away demand from the silver commodity
market.

\newpage

\hypertarget{conclusion}{%
\section{Conclusion}\label{conclusion}}

In this article, I validate how long-term risk-free rates and inflation
are determinants of the silver spot price. More specifically, empirical
analysis suggests that the silver spot price is roughly \emph{twice} as
sensitive to changes in USD expected inflation than it is to changes in
the 10-year U.S. Treasury rate.

When it comes to silver market-timing, it can be concluded the
following:

\textbf{A Strong Bullish \texttt{XAG/USD} Case}

\begin{itemize}
\tightlist
\item
  A silver indicator value \(I^S_t \geq 10\)
\item
  An \emph{unexpected} rise in inflation\footnote{Note that a rise in
    \emph{expected inflation} is incorporated in the silver indicator
    \(I^S_t\)}
\item
  A \emph{future} decrease in long-term risk-free rate\footnote{Note
    that a decrease in the long-term risk-free rate is incorporated in
    the silver indicator \(I^S_t\)}
\item
  An \emph{increase} in demand for silver as a resource for production
\item
  A \emph{decrease} in trust in cryptocurrency
\end{itemize}

\textbf{A Strong Bearish \texttt{XAG/USD} Case}

\begin{itemize}
\tightlist
\item
  A silver indicator value \(I^S_t \leq -10\)
\item
  An \emph{unexpected} decrease in inflation\footnote{Note that a
    decrease in \emph{expected inflation} is incorporated in the silver
    indicator \(I^S_t\)}
\item
  A \emph{future} increase in long-term risk-free rate\footnote{Note
    that a rise in the long-term risk-free rate is incorporated in the
    silver indicator \(I^S_t\)}
\item
  A \emph{decrease} in demand for silver as a resource for production
\item
  A continued \emph{increase} in trust and acceptance in cryptocurrency
\end{itemize}

\hypertarget{appendix-zoomed-plots}{%
\section*{Appendix: Zoomed Plots}\label{appendix-zoomed-plots}}
\addcontentsline{toc}{section}{Appendix: Zoomed Plots}

\includegraphics{silver-market-timing_files/figure-latex/highlighted-plot2-1.pdf}

\includegraphics{silver-market-timing_files/figure-latex/highlighted-plot3-1.pdf}

\includegraphics{silver-market-timing_files/figure-latex/highlighted-plot5-1.pdf}

\newpage

\hypertarget{disclaimer}{%
\section*{Disclaimer}\label{disclaimer}}
\addcontentsline{toc}{section}{Disclaimer}

This article has been prepared and issued by Christian Satzky (CS).

\emph{Accuracy of content:} All information used in the publication of
this report has been compiled from publicly available sources that are
believed to be reliable, however I do not guarantee the accuracy or
completeness of this report and have not sought for this information to
be independently verified. Opinions contained in this report represent
those of CS at the time of publication. Forward-looking information or
statements in this report contain information that is based on
assumptions, forecasts of future results, estimates of amounts not yet
determinable, and therefore involve known and unknown risks,
uncertainties and other factors which may cause the actual results,
performance or achievements of their subject matter to be materially
different from current expectations.

\emph{Exclusion of Liability:} To the fullest extent allowed by law, CS
shall not be liable for any direct, indirect or consequential losses,
loss of profits, damages, costs or expenses incurred or suffered by you
arising out or in connection with the access to, use of or reliance on
any information contained on this note. This article does not constitute
investment advice.

\emph{No investment advice:} The information that I provide should not
be considered in any manner whatsoever as investment advice. Also, the
information provided by me should not be construed by any reader to
effect, or attempt to effect, any transaction in a security.

\emph{Investment in securities mentioned:} CS does not himself hold any
positions in the securities mentioned in this report.

\emph{Copyright:} Copyright 2021 Christian Satzky (CS). No further
distribution of this work is permitted without Christian Satzky's
express written consent.

\newpage

\hypertarget{references}{%
\section*{References}\label{references}}
\addcontentsline{toc}{section}{References}

\hypertarget{refs}{}
\leavevmode\hypertarget{ref-zoo}{}%
Achim Zeileis. 2021. \emph{Package ``Zoo''}.
\url{https://cran.r-project.org/web/packages/zoo/}.

\leavevmode\hypertarget{ref-ggthemes}{}%
Arnold, J. B. et al. 2019. \emph{Package ``Ggthemes''}.
\url{https://cran.r-project.org/web/packages/ggthemes/ggthemes.pdf}.

\leavevmode\hypertarget{ref-gridExtra}{}%
Auguie, B. 2017. \emph{Package ``gridExtra''}.
\url{https://cran.r-project.org/web/packages/gridExtra/gridExtra.pdf}.

\leavevmode\hypertarget{ref-FedInt}{}%
Board of Governors of the Federal Reserve System (US). n.d.
\emph{10-Year Treasury Constant Maturity Rate {[}Dgs10{]}}. Federal
Reserve Bank of St. Louis.
\url{https://fred.stlouisfed.org/series/DGS10}.

\leavevmode\hypertarget{ref-dt}{}%
Dowle, M. et al. 2020. \emph{Package ``Data.table''}.
\url{https://cran.r-project.org/web/packages/data.table/data.table.pdf}.

\leavevmode\hypertarget{ref-FedExpInfl}{}%
FRED. n.d. \emph{10-Year Breakeven Inflation Rate {[}T5yie{]}}. Federal
Reserve Bank of St. Louis.
\url{https://fred.stlouisfed.org/series/T5YIE}.

\leavevmode\hypertarget{ref-stringr}{}%
Hadley Wickham. 2019. \emph{Package ``Stringr''}.
\url{https://cran.r-project.org/web/packages/stringr/}.

\leavevmode\hypertarget{ref-scales}{}%
Hadley Wickham, Dana Seidel, RStudio. 2020. \emph{Package ``Scales''}.
\url{https://cran.r-project.org/web/packages/scales/}.

\leavevmode\hypertarget{ref-kableExtra}{}%
Hao Zhu. 2021. \emph{Package ``kableExtra''}.
\url{https://cran.r-project.org/web/packages/kableExtra/}.

\leavevmode\hypertarget{ref-matrixStats}{}%
Henrik Bengtsson. 2021. \emph{Package ``matrixStats''}.
\url{https://cran.r-project.org/web/packages/matrixStats/}.

\leavevmode\hypertarget{ref-iriz}{}%
Irizarry, Rafael A. 2020. \emph{Introduction to Data Science}.
\url{https://rafalab.github.io/dsbook/}.

\leavevmode\hypertarget{ref-caTools}{}%
Jarek Tuszynski. 2021. \emph{Package ``caTools''}.
\url{https://cran.r-project.org/web/packages/caTools/}.

\leavevmode\hypertarget{ref-xts}{}%
Jeffrey A. Ryan. 2020. \emph{Package ``Xts''}.
\url{https://cran.r-project.org/web/packages/xts/}.

\leavevmode\hypertarget{ref-RcppRoll}{}%
Kevin Ushey. 2018. \emph{Package ``Rcpproll''}.
\url{https://cran.r-project.org/web/packages/RcppRoll/}.

\leavevmode\hypertarget{ref-R-base}{}%
R Core Team. 2020. \emph{R: A Language and Environment for Statistical
Computing}. Vienna, Austria: R Foundation for Statistical Computing.
\url{https://www.R-project.org}.

\leavevmode\hypertarget{ref-fasttime}{}%
Simon Urbanek. 2016. \emph{Package ``Fasttime''}.
\url{https://cran.r-project.org/web/packages/fasttime/}.

\leavevmode\hypertarget{ref-lubridate}{}%
Spinu, V. 2020. \emph{Package ``Lubridate''}.
\url{https://cran.r-project.org/web/packages/lubridate/lubridate.pdf}.

\leavevmode\hypertarget{ref-tidyverse}{}%
Wickham, H. 2019. \emph{Package ``Tidyverse''}.
\url{https://cran.r-project.org/web/packages/tidyverse/tidyverse.pdf}.

\leavevmode\hypertarget{ref-extrafont}{}%
Winston Chang. 2014. \emph{Package ``Extrafont''}.
\url{https://cran.r-project.org/web/packages/extrafont/}.

\leavevmode\hypertarget{ref-knitr}{}%
Xie, Y. et al. 2020. \emph{Package ``Knitr''}.
\url{https://cran.r-project.org/web/packages/knitr/knitr.pdf}.

\end{document}
